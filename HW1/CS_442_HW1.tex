\documentclass[12pt,letterpaper]{article}
\usepackage{fullpage}
\usepackage[top=2cm, bottom=4.5cm, left=2.5cm, right=2.5cm]{geometry}
\usepackage{amsmath,amsthm,amsfonts,amssymb,amscd}
\usepackage{lastpage}
\usepackage{enumerate}
\usepackage{fancyhdr}
\usepackage{mathrsfs}
\usepackage{xcolor}
\usepackage{graphicx}
\usepackage{listings}
\usepackage{hyperref}

\usepackage{xcolor}
\usepackage{listings}
\lstset{basicstyle=\ttfamily,
  showstringspaces=false,
  commentstyle=\color{red},
  keywordstyle=\color{blue}
}

\hypersetup{%
  colorlinks=true,
  linkcolor=blue,
  linkbordercolor={0 0 1}
}
 
\renewcommand\lstlistingname{Algorithm}
\renewcommand\lstlistlistingname{Algorithms}
\def\lstlistingautorefname{Alg.}

\setlength{\parindent}{0.0in}
\setlength{\parskip}{0.05in}

% Edit these as appropriate
\newcommand\course{CS 442}
\newcommand\hwnumber{1}                  % <-- homework number
\newcommand\NetIDa{Boran YILDIRIM}           % <-- NetID of person #1
\newcommand\NetIDb{21401947}           % <-- NetID of person #2 (Comment this line out for problem sets)

\pagestyle{fancyplain}
\headheight 35pt
\lhead{\NetIDa}
\lhead{\NetIDa\\\NetIDb}                 % <-- Comment this line out for problem sets (make sure you are person #1)
\chead{\textbf{\Large Homework \hwnumber}}
\rhead{\course \\ \today}
\lfoot{}
\cfoot{}
\rfoot{\small\thepage}
\headsep 1.5em

\begin{document}

\section*{Question 1}

\textbf{\textit{What is the difference between authentication and authorization?}}\\

First define the meanings of two terms:

\textbf{Authentication}: The process or action of verifying the identity of a user or process \cite{authentication}.

\textbf{Authorization}: A document giving official permission \cite{authorization}. \\

Authentication is done to identify users of the system by using credentials such as email, username and/or password. For example, when a person opens \textit{facebook.com}, system asks his/her email address and password to identify the user. Once user enters his/her credentials correct, \textit{facebook.com} confirms his/her identity and shows the profile or homepage belongs to him/her. The main purpose of authentication is identifying who you are.

Authorization is done to determine users access level after the authentication process is successfully operated. For example, after a user successfully enters \textit{facebook.com}, s/he may choose to upload new photos or delete existing ones of his/her photos. S/he also sees other people's photos, which stands for \textit{read} access. However, s/he cannot delete someone else's photo since s/he is not authorized to \textit{write} to other people's database information.

To conclude, authentication is about who you are and authorization is about what you can \textit{read/write}.

\section*{Question 2}
\textbf{\textit{What is public-key cryptography?}}

Public-key cryptography, aka asymmetric cryptography, is a cryptographic system that uses pairs of keys \cite{publickey}. One of these keys is public and other is private. Public-key can be distributed publicly without considering any security issue since encrypted message with public key can just be decrypted by private-key. On the other hand, private-key must be kept privately to secure the system. This encryption work as one way so if someone knows your public-key, s/he cannot derive private-key.

\newpage

\section*{Question 3}
\textbf{\textit{Describe the steps needed to login using public key cryptography.}}

\begin{enumerate}
\item New key pairs for ssh authentication are created by \textit{ssh-keygen} which is a tool for creating new authentication key pairs for ssh \cite{ssh-keygen}. \textit{Public} and \textit{private} keys are created with \textit{ssh-keygen}.
\begin{lstlisting}[language=bash]
ssh-keygen -t rsa -b 4096
\end{lstlisting}

Private key file \textit{id\_rsa} and public key file \textit{id\_rsa.pub} are generated by RSA algorithm \cite{rsa} using 4096 bits.

\item After login with password to the ssh server, \textit{authorized\_keys list} is added to server by using commands below. 
\begin{lstlisting}[language=bash]
mkdir -p ~/.ssh
touch ~/.ssh/authorized_keys
\end{lstlisting}
By using \textit{ssh-copy-id} generated ssh key is installed to server as an authorized key. 
\begin{lstlisting}[language=bash]
ssh-copy-id -i ~/.ssh/id_rsa.pub user@host
\end{lstlisting}
This logs into the server host, and copies the public key to the server, and configures them to grant access by adding them to the \textit{authorized\_keys list} \cite{ssh-copyid}. Just public key is copied to the server, private key \textit{id\_rsa} is kept secretly on local machine.

\item After logout and login back to system as before, system authenticates \textit{user@host} without asking password. It uses the generated \textit{private key} on local machine.

\end{enumerate}

To identify the user, ssh server encrypts a file with \textit{public key} to challenge the client. If the client has the corresponding \textit{private key}, it will decrypt the message and show it owns the associated \textit{private key}. Then, server can setup the connection for the client.

\newpage

\begin{thebibliography}{9}

\bibitem{authentication} 
Authentication dictionary meaning
\\\texttt{https://en.oxforddictionaries.com/definition/authentication}

\bibitem{authorization} 
Authorization dictionary meaning
\\\texttt{https://en.oxforddictionaries.com/definition/authorization}

\bibitem{publickey} 
Public-key cryptography
\\\texttt{http://en.0wikipedia.org/wiki/Public-key\_cryptography}

\bibitem{ssh-keygen} 
ssh-keygen
\\\texttt{https://www.ssh.com/ssh/keygen/}

\bibitem{rsa} 
RSA algorithm
\\\texttt{https://en.0wikipedia.org/wiki/RSA\_(cryptosystem)}

\bibitem{ssh-copyid} 
ssh-copy-id
\\\texttt{https://www.ssh.com/ssh/copy-id}

\end{thebibliography}

\end{document}
